%%%%%%%%%%%%%%%%%%%%%%% Page Dimension %%%%%%%%%%%%%%%%%%%%%
\usepackage{geometry}
\geometry{a4paper}
\geometry{left=1cm, right=1cm, top=2cm, bottom=2cm}


%%%%%%%%%%%%%%%%%%%%%% Chinese Setting %%%%%%%%%%%%%%%%%%%%%%
\iftoggle{UseChinese}{
  \usepackage{fontspec}
  \usepackage{xeCJK}                        % 讓中英文字體分開設置
  \defaultCJKfontfeatures{AutoFakeBold=2,
                          AutoFakeSlant=.2} % Bold/italic setting.
  \XeTeXlinebreaklocale "zh"                % For auto newline.
  \XeTeXlinebreakskip = 0pt plus 1pt        % For auto newline.
  \newCJKfontfamily{\Kai}{標楷體}
  \newCJKfontfamily{\Hei}{微軟正黑體}
  \newCJKfontfamily{\Wen}{文泉驛等寬微米黑}
  \setCJKmainfont{文泉驛等寬微米黑}
}{}


%%%%%%%%%%%%%%%%%%%%%%%%%%% Indent %%%%%%%%%%%%%%%%%%%%%%%%%
\linespread{0.8}
\setlength{\parskip  }{0.5em}
\setlength{\parindent}{  0em}


%%%%%%%%%%%%%%%%%%%%%% Header And Footer %%%%%%%%%%%%%%%%%%%
\usepackage{fancyhdr}
\usepackage{lastpage}
\pagestyle{fancy}        % plain / empty / headings / myheadings / fancy
\pagenumbering{arabic}   % arabic / roman / Roman / alph / Alph
\fancyhead[L]{}
\fancyhead[C]{223 - 摳的不可}
\fancyhead[R]{}
\fancyfoot[L]{}
\fancyfoot[C]{Page~\thepage ~of~\pageref{LastPage}}
\fancyfoot[R]{}
\renewcommand{\headrulewidth}{0.5pt}
\renewcommand{\footrulewidth}{  0pt}


%%%%%%%%%%%%%%%%%% title format & spacing %%%%%%%%%%%%%%%%%%
\usepackage[sf]{titlesec}
\titleformat{\part}[hang]{\LARGE\bfseries\center}{}{0em}{}[]
\titleformat{\chapter}[hang]{\LARGE\bfseries\center}{}{0em}{}[]
\titleformat{\section}[hang]{\large\bfseries}{}{0em}{}[]
\titleformat{\subsection}[hang]{}{\thesubsection}{1em}{}{}
\titleformat{\subsubsection}[hang]{}{\thesubsubsection}{1em}{}{}

% \arabic 1, 2, 3 ...
% \alph a, b, c ...
% \Alph A, B, C ...
% \roman  i, ii, iii ...
% \Roman  I, II, III ...
% \fnsymbol Aimed at footnotes; prints a sequence of symbols.
\renewcommand{\thesection}{\arabic{section}}
\renewcommand{\thesubsection}{\arabic{section}.\arabic{subsection}}
\renewcommand{\thesubsubsection}
             {\arabic{subsection}.\arabic{subsection}.\arabic{subsubsection}}

% -1 part, 0~5 ch, se, subse, subsubse, par, subpar
\setcounter{secnumdepth}{5}


%%%%%%%%%%%%%%%%%%%%%%% multi columns %%%%%%%%%%%%%%%%%%%%%%
\usepackage{multicol}
\setlength{\columnsep    }{30pt}
\setlength{\columnseprule}{0.5pt}


%%%%%%%%%%%%%%%%%%%%%%% table of content %%%%%%%%%%%%%%%%%%%
\iftoggle{UseTableOfContent}{
  \usepackage[titles]{tocloft}
  \renewcommand{\cftsecleader}{\cftdotfill{\cftdotsep}}
}{}


%%%%%%%%%%%%%%%%%%%%%%%% mathematics %%%%%%%%%%%%%%%%%%%%%%%
\iftoggle{UseMath}{
  \usepackage{amsmath, amsthm, amssymb, mathtools}

  \theoremstyle{definition}
  \newtheorem*{defi}{Definition}

  \theoremstyle{plain}
  \newtheorem{thm}{Theorem}
  \newtheorem{cor}[thm]{Corollary}
  \newtheorem{lemma}[thm]{Lemma}

  \DeclareMathOperator*{\argmin}{argmin}
  \DeclarePairedDelimiter{\ceil}{\lceil}{\rceil}
}{}


%%%%%%%%%%%%%%%%%%%%%%%% algorithm %%%%%%%%%%%%%%%%%%%%%%%%%
\iftoggle{UseAlgorithm}{
  \usepackage{algorithmicx}
  \usepackage{algorithm}
  \usepackage{algc}
  \usepackage{algcompatible}
  \usepackage[noend]{algpseudocode}

  \newcommand{\LineIf}[2]{\State \algorithmicif\ {#1}\ \algorithmicthen\ {#2}}
}{}


%%%%%%%%%%%%%%%%%%%%%%%%%% url, ref %%%%%%%%%%%%%%%%%%%%%%%%
\iftoggle{UseUrl}{
  \usepackage{url}
  \usepackage{hyperref}
  \hypersetup{
    bookmarks   = true,
    unicode     = true,
    colorlinks  = true,
    linkcolor   = blue,
    urlcolor    = blue,
    citecolor   = blue,
    anchorcolor = blue
  }
}{}

%%%%%%%%%%%%%%%%%%%%%%%%%%% Mint %%%%%%%%%%%%%%%%%%%%%%%%%%%
\iftoggle{UseListing}{
  \usepackage{listings}

  \lstset{
    basicstyle=\ttfamily\footnotesize,
    breakatwhitespace=true,
    breaklines=true,
    captionpos=b,
    escapeinside={(*@}{@*)},
    extendedchars=true,
    frame=single,
    keepspaces=true,
    numbers=left,
    numbersep=4pt,
    showspaces=false,
    showstringspaces=false,
    showtabs=false,
    stepnumber=1,
    tabsize=2,
  }
}{}


%%%%%%%%%%%%%%%%%%%%%%%%%%% Mint %%%%%%%%%%%%%%%%%%%%%%%%%%%
\iftoggle{UseMint}{
  \usepackage{listing}
  \usepackage{minted}
  \makeatletter

  \renewcommand\mint[3][]{%
    \DefineShortVerb{#3}%
    \minted@resetoptions
    \setkeys{minted@opt}{#1}%
    \SaveVerb[aftersave={%
      \UndefineShortVerb{#3}%
      \minted@savecode{\FV@SV@minted@verb}%
      \minted@pygmentize{#2}%
      \DeleteFile{\jobname.pyg}}]{minted@verb}#3}
  \renewcommand\minted@savecode[1]{%
    \immediate\openout\minted@code\jobname.pyg\relax
    \immediate\write\minted@code{#1}%
    \immediate\closeout\minted@code}
  \renewcommand\minted@pygmentize[2][\jobname.pyg]{%
    \def\minted@cmd{pygmentize -l #2 -f latex -F tokenmerge
      \minted@opt{gobble} \minted@opt{texcl} \minted@opt{mathescape}
      \minted@opt{linenos} -P "verboptions=\minted@opt{extra}"
      -o \jobname.out.pyg #1}%
    \immediate\write18{\minted@cmd}%
    \ifthenelse{\equal{\minted@opt@bgcolor}{}}%
     {}%
     {\begin{minted@colorbg}{\minted@opt@bgcolor}}%
    \input{\jobname.out.pyg}%
    \ifthenelse{\equal{\minted@opt@bgcolor}{}}%
     {}%
     {\end{minted@colorbg}}%
    \DeleteFile{\jobname.out.pyg}}
  \makeatother

  \newcommand\mintinline[2]{%
    {%
      \RecustomVerbatimEnvironment{Verbatim}{BVerbatim}{}%
      \mint{#1}|#2|
  }}
}{}


%%%%%%%%%%%%%%%%%%%%%%%%%% graphic %%%%%%%%%%%%%%%%%%%%%%%%%
\iftoggle{UseGraphic}{
  \usepackage{graphicx}
}{}


%%%%%%%%%%%%%%%%%%%%%%%%%%% Drawing %%%%%%%%%%%%%%%%%%%%%%
\iftoggle{UseDrawing}{
  \usepackage{pgfplots}

  \usepackage{tikz}
  \usetikzlibrary{shapes}
  \usetikzlibrary{arrows}
  \usetikzlibrary{automata}
  \usetikzlibrary{circuits}
  \usetikzlibrary{circuits.logic.IEC}
  \usetikzlibrary{circuits.logic.US}
}{}


%%%%%%%%%%%%%%%%%%%%%%%% enumerate %%%%%%%%%%%%%%%%%%%%%%%%%
\iftoggle{UseEnumerate}{
  \usepackage{enumitem}
  \setenumerate{label=\textbf{\arabic*.},itemsep=3pt,topsep=3pt}
  \setlist[description]{leftmargin=0pt,labelindent=0pt}
}{}

%%%%%%%%%%%%%%%%%%%%%% custom commands %%%%%%%%%%%%%%%%%%%%%
\newcommand{\Indent}[2]{\parindent=#1 \leftskip=#2}
\newcommand{\Usesize}[2]{\fontsize{#1}{#2}\selectfont\fontsize{#1}{#2}\selectfont}
\newcommand{\Problem}[1]{Problem {#1}}
\newcommand{\SimpleTitle}[1]{\textbf{\Large{\begin{center}#1\end{center}}}}


%%%%%%%%%%%%%%%%%%%% Examples & Usage %%%%%%%%%%%%%%%%%%%%

% \newtheorem{thm}{Theorm}                     定理
% \newtheorem{thm}{Theorm}[section]            加入節...?
% \newtheorem{lem}[thm]{Lemma}                 和定理一起編號
% \newtheorem*{mainthm}{Main Theorem}          不要編號
% \begin{thm}\label{T:major} ... \ref{T:major} 用label, 引用
% \begin{array}{clr}  ... & ... \\             矩陣用
% \begin{align}

% titleformat{command}[shape]{format}{label}{sep}{before}[after]
%     command: 要定義的標題
%     shape  : 形狀 hang|block|display
%     format : 標題外觀(至中對齊啦, 粗體啦...)
%     label  : 標籤(標號)
%     sep    : 標題標籤與內容之間的間隔
%     befor  : 內容之前
%     after  : 內容之後

% titlespacing{command}{left}{before}{after}
%     command: 要定義的標題
%     left   : 標題左邊偏移
%     before : 與前一段文字間隔
%     after  : 與後一段文字間隔

% ~ &                                             空白 表格中的分隔
% \, \: \; \quad \qquad \hspace{2cm}              小中大空白
% \smallskip \medskip \bigskip \vspace{2cm}       垂直移動3, 6, 12, 自訂
% \hfill, \vfill                                  平均的塞入 水平/垂直 空間
% \dotfill \hrulefill                             塞           點/線   進去
% \tiny \small \large \Large \LARGE \huge \Huge   字體大小
% \emph{} \bf{}                                   斜體, 粗體
% \title{} \author{} \date{} \maketitle           封面用
% $...$ \[...\]                                   數學 隨文/展示 模式
% \textbackslash  $\backslash%                    反除號
% \part \chapter \section \paragraph              section command
% \begin{itemize/enumerate/description} \item     item (\itemsep = 0em 設定間距)

% \usepackage[Sonny]{fncychap}                   漂亮的chapter
% Options: Sonny, Lenny, Glenn, Conny, Rejne, Bjarne, Bjornstrup

% 靠左/右/中對齊(前者不切齊)
% \begin{raggedleft / flushright / center}
% \linespread{0.5}

% 子頁面
% \begin{minipage}[t]{0.40\linewidth}
% \end{minipage}

% 表格
% \begin{tabular}[t / b / c]{r|cc||l}
%     bla & x & y & bla    (\\ or \tabularnewline)
%     \hline
%     bla & x & y & bla    (\\ or \tabularnewline)
% \end{tabular}

% multiple columns
% \begin{multicols}{2}
% \end{multicols}

% 虛擬代碼 Example:
% \begin{algorithm}
% \begin{algorithmic}[1]
%     \caption{My algorithm}\label{euclid}
%     \Function{$BinSearch$}{$record array arr, integer len, integer value$}
%         \State{$left \gets 0, {right} \gets \mathrm{len}$}
%         \State ${ans} \gets -1$
%         \While{ $left < right$ }
%             \State $mid \gets ??$
%             \If{$arr[mid] = value$}
%                 \State $ans \gets mid, left \gets mid+1$
%             \ElsIf{$arr[mid] > value$}
%                 \State{$right \gets mid$}
%             \ElsIf{$arr[mid] < value$}
%                 \State{$left  \gets mid+1$}
%             \EndIf
%         \EndWhile
%         \LineIf{$ans = -1$}{\Return $NOTFOUND$}
%         \State \Return $arr[ans]$
%     \EndFunction
% \end{algorithmic}
% \end{algorithm}
